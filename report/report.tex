\documentclass{report}
\usepackage[utf8]{inputenc}

\usepackage[italian]{babel}
\usepackage{graphicx}
\usepackage{float}
\usepackage{hyperref}
\usepackage[italian]{cleveref}

\title{FARGOAL}
\author{
    Bulgarelli Marco, marco.bulgarelli6@studio.unibo.it \and 
    Ravaioli Alessandro, alessandro.ravaioli8@studio.unibo.it \and
    Tassinari Sabrina, sabrina.tassinari@studio.unibo.it \and
    Tramonti Daniele, daniele.tramonti2@studio.unibo.it 
}
\date{15 febbraio 2025}

\begin{document}
\maketitle

\tableofcontents

\chapter{Analisi}

\section{Descrizione e requisiti}

Il software mira alla costruzione di un videogioco ispirato a “Sword of Fargoal” \footnote{
    Il videogioco è stato realizzato da Epyx nel 1982 per VIC-20, l'anno dopo è stato rilasciato per Commodore 64.
}. 
%
Quest’ultimo è un “Dungeon Crawler Arcade”, ovvero basato sull’esplorazione di un labirinto a più piani, con lo scopo di riportare in superficie la Spada di Fargoal attraversando innumerevoli pericoli. 
%
La nostra versione cercherà di essere il più fedele possibile al gioco originale.

\subsection{Requisiti funzionali}
\begin{itemize}
    \item Il giocatore si muoverà all’interno di una grande stanza che corrisponde ad un piano del Dungeon. La generazione di ogni piano (e dei suoi contenuti) dovrà essere casuale.
    \item Il piano è caratterizzato dalla presenza di mostri. Questi possono apparire in luoghi specifici o in punti casuali, aumentando gradualmente con il passare del tempo, rendendo l’ambiente sempre più pericoloso.
    \item All’interno del piano saranno presenti due tipologie di oggetti di cui il giocatore potrà usufruire: dei bauli, che possono contenere degli oggetti magici o delle trappole, oppure delle sacche di monete.
    \item Il sistema di progresso del personaggio è legato all’accumulo di esperienza, che contribuisce all’aumentare del suo livello. Questa può essere ottenuta sia combattendo contro i mostri che offrendo donazioni in oro ai templi.
    \item In ogni piano sarà presente almeno un tempio, dove è possibile donare oro per salire di livello, rigenerare punti ferita e, ogni volta che sono state donate 2000 monete, di essere curati completamente. Inoltre, i templi garantiscono temporanea inattaccabilità ed una rigenerazione accelerata.
    \item Una volta recuperata la spada, è necessario riportarla in superficie, ritornando al primo piano e salendo le scale per uscire dal labirinto. La sfida sta nel fatto che, appena presa in mano la spada, partirà un timer di 15 minuti per ritornare in superficie, se il timer scade la partita è persa.
\end{itemize}

\newpage
\subsection{Modello del dominio}

Il labirinto (\textit{dungeon}) è formato da più piani (\textit{Floor}). 
%
Per muoversi fra un piano e l’altro si utilizzeranno delle rampe di scale; in uno stesso piano saranno presenti più rampe per risalire o per scendere. 
%
Ogni volta che si entra in un nuovo piano, che sia antecedente o successivo, esso verrà generato casualmente. 
%
All’interno di ogni piano saranno presenti vari elementi (in \ref{img:DiagrammaUMLAnalisi} si chiamano \textit{FloorElement}): alcuni hanno la caratteristica di essere interagibili (in \ref{img:DiagrammaUMLAnalisi} \textit{Interactable}), mentre gli altri elementi sono delle entità (\textit{Entity}), che si muovono all’interno del piano. 
%
I primi, che sono fissi nella mappa, sono l’insieme formato dalle ceste (che contengono oggetti magici), le scale (in \ref{img:DiagrammaUMLAnalisi} \textit{Stairs}) e il tempio (in \ref{img:DiagrammaUMLAnalisi} \textit{Temple}) , che è unico all’interno del piano. 
%
Le entità, invece, sono l’insieme dei mostri (in \ref{img:DiagrammaUMLAnalisi} \textit{Monster}) e del giocatore (in \ref{img:DiagrammaUMLAnalisi} \textit{Player}); entrambi si muovono liberamente all’interno del piano e possono entrare in combattimento l’uno con l’altro. 

\begin{figure}[H]
    \centering
    \includegraphics[width=12cm]{DiagrammaAnalisi.drawio.png}
    \caption{Schema UML dell'analisi del problema, con rappresentate le entità principali ed i rapporti fra di loro}
    \label{img:DiagrammaUMLAnalisi}
\end{figure}

\chapter{Design}

\section{Architettura}

L'architettura dell'applicazione segue il pattern architetturale MVC. Il punto di partenza della nostra applicazione è la classe \textit{GameEngine} che inizializza \textit{View, Model} e \textit{Controller}. 
%
La \textit{View} è il punto d'ingresso degli input, la quale li riceve ed in seguito notifica l'\textit{InputController}. Quest'ultimo grazie all' \textbf{InputComponent} riesce ad aggiornare 
%
il \textit{Model}, che nel nostro caso è la \textbf{SceneManager}. Questa è estesa da \textbf{FloorManager}, interfaccia che si occupa di gestire collisioni e gestire ed aggiornare
%
gli elementi attivi nel piano. Infine utilizzando l'interfaccia \textbf{Renderer}, i vari \textit{FloorElement} vengono aggiunti in coda nella \textit{View} in attesa di essere
%
ridisegnati nel seguente \textbf{update()}.
%

\begin{figure}[H]
    \centering{}
    \includegraphics[width=13cm]{MVCProgetto.png}
    \caption{Architettura dell'applicazione (MVC)}
    \label{img:analysis}
\end{figure}

\section{Design dettagliato}

\subsection{Bulgarelli Marco}

\subsubsection{Organizzazione del Player}

\paragraph{Problema} Fornire interfacce che permettano una gestione più efficiente del Player e migliorare la leggibilità del codice.

\paragraph{Soluzione} Suddivisione della gestione del Player in tre classi specifiche, ognuna con un ruolo ben definito:

\begin{itemize}
    \item \textit{Inventory} per la gestione dell'inventario.
    \item \textit{Gold} per la gestione del denaro.
    \item \textit{Health} per la gestione dei punti vita.
\end{itemize}

La classe \textit{PlayerImpl} si compone di due elementi fondamentali per la supervisione delle risorse del giocatore: \textit{Inventory} e \textit{Gold}. \newline
%
\textit{Inventory}, implementato da \textit{InventoryImpl}, gestisce gli oggetti del giocatore, consentendo di aggiungerli, rimuoverli e monitorarne la quantità
%
all’interno dell’inventario. \textit{Gold}, implementato da \textit{GoldImpl}, gestisce invece il numero di monete trasportate dal Player, il limite massimo trasportabile
%
e le donazioni effettuate al tempio. \newline
%
Separare questa logica dalla classe \textit{PlayerImpl} evita ridondanze e migliora la leggibilità del codice. \newline
%
Oltre a queste componenti, \textit{PlayerImpl} include \textit{Health}, la cui implementazione \textit{HealthImpl} si occupa della gestione dei punti vita del giocatore e
%
delle operazioni correlate. Questa classe è stata progettata a livello di entità (\textit{Entity}) per consentirne il riutilizzo anche da parte dei mostri, garantendo maggiore flessibilità. \newline
%
Grazie a questa suddivisione, il codice risulta più leggibile, modulare ed estensibile, facilitando future modifiche e aggiunte senza compromettere la chiarezza della struttura.

\begin{figure}[H]
    \centering
    \includegraphics[width=13cm]{PlayerImpl.png}
    \caption{Schema UML delle classi che compongono \textit{PlayerImpl} usate per semplificare l'organizzazione del Player.}
    \label{img:PlayerImpl.png}
\end{figure}

\paragraph{Problema} Gestione dell’input da tastiera, evitando di creare dipendenze tra model e view.

\paragraph{Soluzione} Utilizzo del \textit{pattern di progettazione Observer}. \newline
%
L’interfaccia \textit{View} è responsabile della ricezione degli input direttamente dal Client. Una volta acquisiti, questi input vengono trasmessi all’interfaccia \textit{InputController}, 
%
che viene implementata da \textit{KeyboardInputController}. 
%
Il ruolo di \textit{InputController} è quello di fungere da intermediario tra \textit{View} e \textit{InputComponent}, garantendo che le azioni del giocatore vengano interpretate correttamente. \newline 
%
Grazie a \textit{InputController}, il sistema è in grado di rilevare quali tasti vengono premuti e in quale momento, traducendo questi comandi in istruzioni comprensibili per il gioco.
%
Una volta ricevute le informazioni da \textit{InputController}, \textit{InputComponent} le elabora e le utilizza per eseguire le azioni richieste dal giocatore, come il movimento, l’interazione 
%
con l’ambiente o l’utilizzo di oggetti.

\begin{figure}[H]
    \centering
    \includegraphics[width=13cm]{ObserverPattern.png}
    \caption{Diagramma UML che rappresenta il flusso degli input.}
    \label{img:ObserverPattern.png}
\end{figure}

\clearpage
\subsection{Ravaioli Alessandro}

Il mio ruolo principale riguardava la realizzazione della mappa e della struttura che gestisce gli elementi sul piano.
%
Ogni mappa è composta da stanze, ovvero spazi più larghi di caselle camminabili, e corridoi, delle sequenze di caselle che possono collegare due stanze tra di loro.
%

\subsubsection{Generazione mappa del piano}

\paragraph{Problema} Bisogna generare la mappa del piano con possibilmente una sembianza di difficoltà incrementale

\begin{figure}[H]
    \centering
    \includegraphics[width=9cm]{mapUMLdiagram.png}
    \caption{Schema UML della generazione della mappa tramite builder.}
    \label{img:mapBuilder}
\end{figure}

\paragraph{Soluzione} Dato che la generazione della mappa è definita da dei passaggi sequenziali e ben precisi, ho deciso che il miglior modo per poterla costruire fosse con un builder.
%
Per evitare che venisse chiamata la costruzione fuori ordine l’ho realizzata come classe interna a FloorGenerator, questo permette anche di semplificare la creazione della mappa dall’esterno.
%
Per riuscire a dare un'impressione di difficoltà della mappa ho fatto in modo che vengano generate più stanze e corridoi ai piani bassi e che ne vengano invece generate di meno scendendo più a fondo, 
%
questo permette anche di ridurre la possibilità di rimanere bloccati ai primi piani quando non si hanno le risorse per evitarlo.

\clearpage
\subsubsection{Strutturare i gestori}

\paragraph{Problema} Strutturare i gestori in modo da minimizzare il numero di modifiche per crearne nuovi o modificarli.

\begin{figure}[H]
    \centering
    \includegraphics[width=9cm]{SceneStructureDiagram.png}
    \caption{Schema UML della struttura degli SceneManager.}
    \label{img:SceneStructure}
\end{figure}

\paragraph{Soluzione} Ho modellato le scene con interfacce e classi astratte evitando il più possibile ripetizioni.
%
Sono arrivato a una struttura dove posso definire le basi dei manager nell'interfaccia SceneManager e specializzarli poi in due categorie.
%
Di queste due categorie sono riuscito a utilizzare il \textit{Pattern Template Method} per definire tutte le funzioni dei MenuManager data la grande somiglianza tra le loro funzioni.

\clearpage
\subsection{Tassinari Sabrina}

\subsubsection{Generazione di item in un forziere}

\paragraph{Problema} Il giocatore, esplorando, può trovare dei forzieri, con al loro interno degli items utili oppure delle trappole. 

\paragraph{Soluzione} Per gestire la generazione di item all'apertura della cesta si utilizza il \textit{pattern Abstract Factory}.
%
L'interfaccia creatrice fornisce i metodi factory che hanno il compito di creare tutti gli item che si possono trovare nei forzieri, come mostrato nella figura \ref{img:chestItemFactory}. 
%
Quando il giocatore interagisce con il forziere utilizzando il metodo \textbf{interact()}, usando il metodo opportuno, viene generato un item scelto in modo randomico tra tutti.

\begin{figure}[H]
    \centering
    \includegraphics[width=7cm]{patternFactory.drawio.png}
    \caption{Schema UML dell'applicazione del \textit{pattern AbstractFactory} e di dove viene usata la Factory.}
    \label{img:chestItemFactory}
\end{figure}

\clearpage
\subsubsection{Riuso del codice per un determinato tipo di item}

\paragraph{Problema} Sviluppando i vari item che il giocatore può trovare nei forzieri, ci si è accorti che quelli dello stesso tipo, quindi incantesimi, trappole e utility, condividono molte caratteristiche.
%
Questo porta a classi molto simili, con metodi ripetuti. 

\paragraph{Soluzione} Ogni tipo di item ha delle caratteristiche simili; è stato quindi utilizzato il \textit{pattern Template Method}, in modo da poter riutilizzare la stessa classe per sviluppare più item, come da \ref{img:templateItem}.
%
Per ogni tipo di item il metodo template è \textbf{use()}, che utilizza il metodo astratto \textbf{effect()}. 

\begin{figure}[H]
    \centering
    \includegraphics[width=7cm]{patternTemplateItem.drawio.png}
    \caption{Schema UML dell'applicazione del \textit{pattern Template Method}. Si è usato l'esempio del DriftSpell, della CeilingTrap e del Beacon, ma le classi che estendono le classi astratte sono di più.}
    \label{img:templateItem}
\end{figure}

\clearpage
\subsection{Tramonti Daniele}

\subsubsection{Struttura base dei nemici}

\paragraph{Problema} I nemici presenti nel gioco possono essere diversi tra loro, ma presentano tutti la stessa struttura di base che comprende per esempio:

\begin{itemize}
    \item salute
    \item skill
    \item una posizione nella mappa
    \item un livello
    \item un tag che ne identifica il tipo
    \item un campo per definire la visibilità
    \item un campo per definire se il mostro è in un combattimento
\end{itemize}

\paragraph{Soluzione} Ho deciso di utilizzare un un'unica implementazione di una classe astratta \textit{AbstractMonster} che contiene già tutte le caratteristiche di base di un mostro, il quale poi nella sua classe
%
specifica, che estende la classe generale \textit{AbstractMonster}, potrà ricevere qualità personalizzate. La scelta di questa soluzione è stata fatta per permettere:

\begin{itemize}
    \item \textbf{estendibilità:} grazie a questa struttura l'aggiunta di un nuovo mostro è molto semplice;
    \item \textbf{riuso:} con l'utilizzo della classe astratta generale ogni mostro ha la stessa implementazione di base senza però dover duplicare codice per ognuno.
\end{itemize}

\begin{figure}[H]
    \centering
    \includegraphics[width=6cm]{AbstractMonster.png}
    \caption{Schema UML dell'uso dell'\textit{Abstract class AbstractMonster}. Si è usato l'esempio del Rogue, del Barbarian e del Mage, ma le classi che la estendono sono di più.}
    \label{img:AbstractMonster}
\end{figure}

\subsubsection{Creazione dei nemici}

\paragraph{Problema} Ogni tipologia di nemico nel gioco ha un preciso piano dove può esistere e compiere azioni, perciò far apparire il mostro corretto nel piano in cui si trova l'avventuriero è
%
tutt'altro che un problema immediato da risolvere.

\paragraph{Soluzione} Per riuscire a trovare una soluzione a questo problema, ho utilizzato il \textit{pattern creazionale Factory Method} per far sì che fosse presente una factory generale
%
con un singolo metodo pubblico da poter essere richiamato. Nella sua implementazione poi sono stati creati dei metodi per generare ogni tipo di mostro che sarebbero stati chiamati, in base
%
al piano in cui il player si sarebbe trovato in quel momento, dal metodo generale. Questa soluzione è stata attuata per:

\begin{itemize}
    \item \textbf{facilità nell'uso:} basta richiamare il singolo metodo \textbf{generate()} della factory e verrà creato in automatico il mostro corretto per quel piano;
    \item \textbf{estendibilità:} è possibile aggiungere un metodo specifico nell'implementazione della \textit{factory} per aggiungere un nuovo mostro;
    \item \textbf{divisione delle responsabilità:} viene delegato ad una componenente specializzata un compito di cui non si devono interessare altre classi.
\end{itemize}

\begin{figure}[H]
    \centering
    \includegraphics[width=9cm]{FactoryMethod.png}
    \caption{Schema UML dell'applicazione del \textit{pattern creazionale Factory Method}.}
\end{figure}

\clearpage
\subsubsection{Movimento dei nemici}

\paragraph{Problema} Ogni nemico si deve muovere nella mappa autonomamente cercando, se possibile, di inseguire ed attaccare l'avventuriero.

\paragraph{Soluzione} Alla fine sono arrivato alla decisione di creare un algoritmo, in modo autonomo, uguale per tutti i mostri con lo scopo di rendere i movimenti il più realistici possibili. Ho 
%
utilizzato una classe statica separata, denominata \textbf{Ai}, il cui metodo \textbf{move()} viene chiamato da ogni mostro quando è il proprio turno di muoversi. L'utilizzo di algoritmi come per esempio
%
\textit{Shortest Path di Dijkstra} sono stati scartati per:

\begin{itemize}
    \item \textbf{tempistiche:} ricalcolare il percorso migliore per ogni singolo mostro ed ad ogni movimento del player (circa al massimo 10 al secondo) avrebbe potuto portare a tempistiche di calcolo
%
non ottimali e di conseguenza ad un gameplay non performante.
\end{itemize}

%
La scelta di questa soluzione è stata fatta per permettere:

\begin{itemize}
    \item \textbf{facilità nell'uso:} basta richiamare il metodo statico \textbf{move()} della classe \textit{Ai} per ottenere il movimento del mostro;
    \item \textbf{divisione delle responsabilità:} viene delegato ad una classe secondaria un compito per non farlo svolgere direttamente alla classe del mostro.
\end{itemize}

\begin{figure}[H]
    \centering
    \includegraphics[width=9cm]{MonsterMovement.png}
    \caption{Scherma UML dell'applicazione della \textit{classe statica Ai}. Si è usato l'esempio del Rogue, del Barbarian e del Mage, ma tutte le classi dei mostri fanno riferimento ad \textit{Ai}.}
\end{figure}

\chapter{Sviluppo}

\subsection{Testing automatizzato}
Per testare l'applicazione abbiamo fatto uso di diversi test che testavano i vari elementi del Model:
\begin{itemize}
    \item \textbf{Inventario}, del quale si testa l’inizializzazione e la gestione degli incantesimi.
    \item \textbf{Oro}, del quale si testa: l’inizializzazione, la gestione delle operazioni di acquisizione, perdita e reset dell’oro, il metodo per modificare la capacità massima di oro trasportabile e la meccanica della donazione dell’oro al tempio.
    \item \textbf{Mostri}, dei quali viene testata la generazione idonea nel piano corrispondente, il movimento corretto nelle situazioni in cui si possono trovare e il giusto funzionamento dei danni verso il Player.
    \item \textbf{Interactables}, dei quali viene testato se una volta trovati vengono effettivamente aggiunti nell'inventario e se una volta usati abbiano l'effetto desiderato sul Player.
    \item \textbf{Mappa}, della quale è stato testato se il metodo che restituisce una casella casuale. restituisse effettivamente una casella casuale. Inoltre è stato testato che le tempistiche di creazione della mappa non fossero eccessive.
    \item \textbf{FloorManager}, del quale è stato testato la corretta inizializzazione di piani, l'effettivo scorrere dei piani e se l'uscita dal \textit{Dungeon} con la spada funzionasse.
    \item \textbf{Player}, del quale si testano una vasta gamma di funzionalità, di seguito illustrate in maggior dettaglio:
\end{itemize} 
%
In merito all'inventario, si verifica che venga inizializzato correttamente con il numero previsto di oggetti, confrontando i valori attesi di pozioni, incantesimi e altri elementi con quelli effettivi. Inoltre, si controlla che 
%
la lista delle mappe dei diversi piani del dungeon non sia nulla. \newline
%
Viene anche effettuato un test sull’incantesimo di invisibilità, utilizzato come campione, per accertarsi che i metodi dedicati alla modifica e alla verifica della sua quantità nell’inventario funzionino correttamente. \newline
%
Relativamente all’oro, si verifica che all’istanziazione il valore iniziale dell’oro corrente sia zero, che la capacità massima di oro trasportabile sia cento e che l’oro donato dal giocatore sia zero. Successivamente, 
%
si controlla che l’oro aggiunto venga accumulato correttamente senza superare la capacità massima. Si testa inoltre che, dopo aver accumulato l’oro, il metodo di reset riporti il valore dell’oro a zero. \newline
%
Viene verificata anche la possibilità di modificare la capacità massima dell’oro, assicurandosi che non si possa superare il limite imposto. Si controlla infine che l’oro donato al tempio possa essere impostato e modificato correttamente,
%
e che la rimozione dell’oro funzioni come previsto, impedendo che il valore dell’oro corrente scenda sotto zero. \newline
%
Per quanto riguarda il Player, si verifica che venga correttamente inizializzato con i valori di default appropriati e che i metodi setter funzionino correttamente, modificando le sue proprietà.
%
Viene inoltre testata la correttezza dei metodi che incrementano i valori del giocatore, come i punti esperienza ed il numero di nemici uccisi. Si controlla anche che il giocatore possa salire di livello correttamente, 
%
aggiornando i punti esperienza e quelli necessari per il livello successivo. Viene eseguito un test per verificare che il giocatore possa muoversi correttamente, aggiornando la sua posizione nel piano. \newline
%
Per quanto riguarda le battaglie, i test sono suddivisi in due scenari: se il giocatore vince lo scontro, si verifica che ottenga le ricompense previste, come i punti esperienza e l’aumento del contatore dei nemici uccisi;
%
se il giocatore perdem si verifica che venga dichiarato morto e che non possa più combattere. Inoltre, si controlla che il giocatore infligga danni correttamente al mostro e che riceva danni in risposta. Si verifica anche 
%
che il giocatore venga dichiarato morto alle condizioni appropriate. \newline
%
Vengono eseguiti due test sulla rigenerazione: uno per verificare che avvenga dopo un certo periodo di tempo in condizioni normali, uno per confermare che avvenga in un quinto del tempo quando l’incantesimo di rigenerazione 
%
è attivo e il giocatore si trova in un tempio. \newline
%
Infine, si verifica che l’utilizzo degli oggetti dell’inventario riduca correttamente la quantità disponibile e che la profondità massima raggiunta dal giocatore nel dungeon venga aggiornata correttamente. \newline
%
Per quanto riguarda gli items si è testato se la loro generazione fosse corretta e se il loro funzionamento era giusto, quindi se quando venivano usati il loro numero decrementava e se quando venivano trovati il loro numero aumentava. \newline
%

\subsection{Note di sviluppo}

\subsubsection{Bulgarelli Marco}

\begin{itemize}
    \item \textbf{Utilizzo di \texttt{lambda expression}:}\newline
    Utilizzate ad esempio in:
    \begin{sloppypar}
        \url{https://github.com/ravag/OOP24-fargoal/blob/d519fa7565af74f3ad12632d480b3a3f8f15e8af/src/main/java/fargoal/model/entity/player/impl/PlayerImpl.java#L441-L445}\newline
        \url{https://github.com/ravag/OOP24-fargoal/blob/d519fa7565af74f3ad12632d480b3a3f8f15e8af/src/main/java/fargoal/model/entity/player/impl/PlayerImpl.java#L577-L580}
    \end{sloppypar}
    \item \textbf{Utilizzo di \texttt{Stream}, di \texttt{Optional} o di altri costrutti funzionali:}\newline
    Utilizzati ad esempio in:
    \begin{sloppypar}
        \url{https://github.com/ravag/OOP24-fargoal/blob/d519fa7565af74f3ad12632d480b3a3f8f15e8af/src/main/java/fargoal/model/entity/player/impl/PlayerImpl.java#L443}\newline
        \url{https://github.com/ravag/OOP24-fargoal/blob/d519fa7565af74f3ad12632d480b3a3f8f15e8af/src/main/java/fargoal/model/entity/player/impl/PlayerImpl.java#L457}\newline
        \url{https://github.com/ravag/OOP24-fargoal/blob/d519fa7565af74f3ad12632d480b3a3f8f15e8af/src/main/java/fargoal/model/entity/player/impl/PlayerImpl.java#L577}
    \end{sloppypar}
\end{itemize}

\subsubsection{Ravaioli Alessandro}
\paragraph{Utilizzo di \texttt{Stream} e \texttt{lambda expression}}
Utilizzati in: 
\begin{sloppypar}
    \url{https://github.com/ravag/OOP24-fargoal/blob/33688c471035d4ce72dd7ede6749bdbd755a3ec9/src/main/java/fargoal/model/map/impl/FloorGeneratorImpl.java#L96-L99}
\end{sloppypar}

\subsubsection{Tassinari Sabrina}
\begin{itemize}
    \item \textbf{Utilizzo di \texttt{Stream} e \texttt{lamba expression}:} \newline
    Esempio di codice dove vengono utilizzati entrambi:
    \begin{sloppypar}
        \url{https://github.com/ravag/OOP24-fargoal/blob/0ffa0e6b567dd64bd80c4c169fa41b394a9408cc/src/main/java/fargoal/model/interactable/pickupable/inside_chest/spell/impl/TeleportSpell.java#L40C9-L42C96}
    \end{sloppypar}
    \item \textbf{Utilizzo di \texttt{Optional}:}\newline
    Esempio di codice dove li ho utilizzati:
    \begin{sloppypar}
        \url{https://github.com/ravag/OOP24-fargoal/blob/0ffa0e6b567dd64bd80c4c169fa41b394a9408cc/src/main/java/fargoal/model/interactable/pickupable/on_ground/SwordOfFargoal.java#L33}
    \end{sloppypar}
\end{itemize}

\subsubsection{Tramonti Daniele}
\begin{itemize}
    \item \textbf{Utilizzo di \texttt{Stream} e \texttt{lambda expression}:}\newline
    Esempio di codice dove vengono utilizzati entrambi:
    \begin{sloppypar}
        \url{https://github.com/ravag/OOP24-fargoal/blob/cb5f09a083bb0696ec37722b48ac348cbd7187d8/src/main/java/fargoal/model/entity/monsters/ai/Ai.java#L109C9-L120C18}
    \end{sloppypar}
\end{itemize}
\paragraph{}

\chapter{Commenti finali}

\subsection{Autovalutazione e lavori futuri}

\subsubsection{Bulgarelli Marco}
Mi sono principalmente occupato della gestione e dell’organizzazione del Player, un compito che, sebbene complesso, mi ha permesso di affrontare numerose sfide. Fortunatamente, ho potuto contare su materiale didattico di alta qualità e sul supporto dei miei compagni, con i quali ho collaborato in maniera armoniosa. \newline
L’esperienza si è rivelata estremamente formativa. Non solo ho avuto l’opportunità di approfondire le mie conoscenze del linguaggio Java, ma ho anche imparato a padroneggiare le basi della programmazione orientata agli oggetti. Ho compreso l’importanza della progettazione accurata che sta alla base di ogni software complesso 
e ho acquisito familiarità con strumenti fondamentali come Git e LaTeX. Questi strumenti non solo saranno utili per i corsi futuri, ma rappresentano anche competenze che mi accompagneranno nel corso della mia carriera professionale. \newline
Questo è stato il mio primo progetto di tale portata e, lavorando in gruppo, ho potuto apprezzare quanto siano essenziali l’organizzazione e la comunicazione tra i membri. Ho imparato che il successo di un team dipende dalla capacità di semplificare il più possibile il lavoro di tutti, affinché ogni componente possa concentrarsi 
sulle proprie competenze e contribuire al meglio. \newline
Mi ritengo molto soddisfatto del risultato ottenuto e sono contento che il gioco rimanga piuttosto fedele all’originale, con il quale sono cresciuto, il ché aggiunge un valore emotivo a questo progetto. Mi piacerebbe proseguire il lavoro, rifinendo e ottimizzando alcune meccaniche relative al Player, ma anche introducendo nuovi 
elementi che possano arricchire ulteriormente l’esperienza di gioco e renderla ancora più coinvolgente.

\subsubsection{Ravaioli Alessandro}
Nel gruppo il mio ruolo era principalmente quello di creare la mappa di gioco e di gestire gestire tutte le cose che ci andavano sopra.
%
Non sono completamente soddisfatto del mio lavoro in quanto mi ero posto come obiettivo extra quello di rendere impossibile il bloccarsi nella mappa, a differenza dell' originale,
%
però sono rimasto soddisfatto del risultato finale e del lavoro dei miei compagni. Questa è stata una buona esperienza di esempio per il programmare in gruppo. 
%
Se tornerò su questo progetto decisamente cambierò la generazione della mappa, anche se non diventa più fedele, per fare in modo che non si rimanga più bloccati.

\subsubsection{Tassinari Sabrina}
Per quanto riguarda la mia parte del lavoro, mi sono occupata degli Interectable, quindi degli oggetti magici con cui il giocatore interagisce nel gioco.
%
Sono rimasta soddisfatta del mio lavoro e di quello dei miei compagni: siamo riusciti ad implementare le funzioni obbligatorie che ci eravamo posti inizialmente e fra di noi abbiamo lavorato bene. 
%
Inoltre, questo progetto mi ha aiutato molto per quanto riguarda la crescità delle mie abilità: precedentemente non mi ero mai cimentata in un progetto di questo tipo e penso di aver imparato molto nell'ambito della programmazione e del lavoro in gruppo.
%
In futuro mi piacerebbe ritornare su questo progetto aggiungendo nuovi oggetti magici e questo mi sarà facile farlo, grazie al modo in cui ho implementato le classi.
%
 

\subsubsection{Tramonti Daniele}
La mia posizione nel progetto è stata quella di creare i mostri e di prendermi cura del loro movimento, facendo sì che cerchino di muoversi verso l'avventuriero per attaccarlo.
%
Questa esperienza, anche se certe volte un po' faticosa, è stata però un bel momento di crescita personale e tecnica verso la programmazione con Java, ma soprattutto è stata un'occasione
%
ottima per sperimentare il lavoro in team, che penso sia una delle principali challenge nell'ambiente lavorativo della programmazione. Sono anche abbastanza soddisfatto del risultato che 
%
rispecchia molto l'idea che ci eravamo posti all'inizio e rimane abbastanza fedele al gioco originale al quale ci siamo ispirati. Sicuramente, se in futuro avrò tempo e occasione, non mi tirerò
%
indietro dall'ottimizzare lo sviluppo del gioco e migliorarlo negli aspetti che troveremo interessanti da aggiungere. Infine sono rimasto molto contento anche del corso di OOP, un corso molto
%
interessante e che, seppur partendo da zero con la programmazione in Java, è stato in grado di accompagnarmi molto bene nell'apprendimento di questo nuovo linguaggio, merito soprattutto ai docenti
%
che sono stati sempre molto professionali, chiari e disponibili per ogni richiesta.

\appendix
\chapter{Guida utente}

Il giocatore, quando aprirà l’applicazione si troverà di fronte la schermata iniziale del gioco. 
%
Si potrà scegliere, premendo la \underline{barra spaziatrice}, tra due opzioni: \textbf{START GAME} oppure \textbf{EXIT}.
%
Per poter iniziare a giocare si scelga con le \underline{frecce ↑ e ↓}  la scritta \textbf{START GAME}, colorandola di \textit{azzurro}, in questo modo si potrà iniziare a giocare.
%
Quando la partità inizia il giocatore si troverà nella mappa del primo livello del dungeon.
%
Noterà che essa sarà quasi tutta oscurata, tranne la parte dove l’avventuriero si trova. 
%
Esso, muovendosi, potrà scoprire altre zone della mappa, che diventerà visibile mano a mano che si esplora il piano.
%
Per muovere l’avventuriero si dovranno premere le frecce ← per andare a sinistra, ↑ per andare verso l’alto, → per andare a destra e ↓ per andare verso il basso.
%
Per \textbf{ingaggiare un combattimento} con i mostri presenti nel piano basterà solo \underline{avvicinarsi} e il combattimento inizierà da solo.
%
Nel piano, inoltre, ci saranno degli oggetti con i quali il giocatore potrà interagire.
%
Per interagire con il \textbf{tempio} il giocatore dovrà solo \underline{posizionarsi sopra quello}, in automatico i soldi verranno donati e il giocatore sarà immune a qualsiasi attacco.
%
Per interagire con altri oggetti, quali \textbf{scale per scendere e salire in piani differenti}, \textbf{sacche di monete} e \textbf{ceste} va invece premuta la \underline{barra spaziatrice}.
%
In particolare i sacchi di monete potrebbero contenere più monete di quelle che si possono portare: i soldi in più verranno sotterrati nel punto dove la sacca d’oro è stata trovata, in attesa che il giocatore abbia abbastanza spazio nell’inventario per riprendere le monete premendo di nuovo la \underline{barra spaziatrice}. 
%
\\
%
Nelle ceste, oltre a trappole che danneggiano il giocatore, verranno trovati degli oggetti che saranno aggiunti direttamente nell’inventario.
%
Gli oggetti in questione sono:
%
\begin{itemize}
    \item \textbf{Drift spell}: questo incantesimo viene usato premendo il \underline{tasto D}. Esso, se attivo, non fa danneggiare il giocatore quando cade in un buco. Quando l'avventuriero apre una cesta, infatti, potrebbe trovare una trappola di nome \textit{pit}, che può essere evitata lanciando questo incantesimo.
    \item \textbf{Invisibility spell}: questo incantesimo viene usato premendo il \underline{tasto I}. Esso rende invisibile il giocatore ai mostri. 
    \item \textbf{Light spell}: questo incantesimo viene usato premendo il \underline{tasto L}. Esso permette di espandere l’area illuminata della mappa. Il giocatore, quindi, scopre aree maggiori andando avanti nella mappa. Quando l’incantesimo è attivato, però, se l’incantesimo dell'invisibilità è attivo il giocatore è visibile ai mostri. Si può accendere o spegnere la luce premendo il tasto O.
    \item \textbf{Regeneration spell}: questo incantesimo viene usato premendo il \underline{tasto R}. Esso rende la rigenerazione della salute dell’avventuriero il doppio più veloce; viene quindi riacquistato un punto ferita ogni 5 secondi; normalmente, infatti, verrebbe riacquisito un punto ferita ogni 10.
    \item \textbf{Shield spell}: questo incantesimo viene usato premendo il \underline{tasto S}. Esso rende invulnerabile il giocatore nello scontro successivo.
    \item \textbf{Teleport spell}: questo incantesimo viene usato cliccando il \underline{tasto T}. Il giocatore viene teletrasportato vicino ad una torcia, cioè un \textit{beacon}, se posizionato precedentemente dal giocatore nella mappa, o in una posizione casuale.
    \item \textbf{Beacon}: questo oggetto viene posizionato a terra premendo il \underline{tasto B}. Quando esso è a terra il giocatore, se vicino ad esso, è invulnerabile.
    \item \textbf{Enchanted weapon}: questo oggetto incrementa la skill del giocatore, incrementa quindi il danno che il giocatore fa in combattimento.
    \item \textbf{Healing potion}: premendo il \underline{tasto H} il giocatore può bere una pozione di cura, che farà recuperare all’avventuriero dei punti ferita. Essa viene bevuta in automatico, a patto che l'avventuriero ne abbia una in inventario, se il giocatore ha tra i -5 e i 0 punti ferita, e sia in combattimento con un mostro.
    \item \textbf{Magic sack}: sono delle sacche nelle quali viene posizionato l’oro che il giocatore trova nel dungeon. Ogni sacca può contenere 100 monete e l’avventuriero, inizialmente, ha solamente una di queste nell’inventario.
    \item \textbf{Map}: il giocatore potrà trovare delle mappe di determinati livelli del dungeon; quando il giocatore arriverà a quel piano la mappa sarà illuminata, non buia.
\end{itemize}
%
Per quanto riguarda gli oggetti che si possono utilizzare premendo un tasto, se non si ha il tipo di oggetto del quale si preme il tasto non succede nulla.
%
\\
%
Il gioco procede fino a che il giocatore non finisce i suoi punti vita (quindi arriva a -5) oppure fino a che non ritorna in superficie con la leggendaria Spada di Fargoal, che appare tra il quindicesimo e il ventesimo piano del dungeon e può essere raccolta premendo la \underline{barra spaziatrice}.
%
Quando il gioco finisce comparirà la schermata di fine partita, dalla quale il giocatore potrà scegliere, sempre con le frecce, se giocare una nuova partita o uscire dall'applicazione.

\end{document}